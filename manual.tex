\documentclass[a4paper]{article}

\usepackage{verbatim}

\newcommand{\PM}{Polymake}

\begin{document}
\section{Installing libsingular}
Install {\tt libreadline-dev} and {\tt cddlib} before.
\begin{verbatim}
git clone https://github.com/Singular/Sources
cd Sources
git checkout master
export CC="gcc -fpic -DPIC -DLIBSINGULAR"
export CXX="g++ -fpic -DPIC -DLIBSINGULAR"
./configure
make install-libsingular
\end{verbatim}
\section{Installing Polymake}

\section{Installing the extension}

\begin{verbatim}
git clone git@github.com:lkastner/polymake_algebra.git
cd polymake_algebra
git checkout master
\end{verbatim}

Start \PM and enter the following command:
\begin{verbatim}
import_extension "directory/polymake_algebra";
\end{verbatim}
\section{Ideals in Polymake}
One can build an ideal from an array of polynomials as follows:
\begin{verbatim}
init_singular("Path to libsingular.so");
$r=new Ring(qw(x y z));
($x,$y,$z)=$r->variables;
$p1=3*$x*$y + 2*$z + 1;
$p2=2*$x + 1/2;
print $p1," , ",$p2,"\n";
$i = new Ideal(GENERATORS=>[$p1,$p2]);
\end{verbatim}
Properties working:
\begin{enumerate}
\item STANDARD: Computes a standard basis of the ideal. So far we assume a global ordering on the ring of polynomials. No other rings or orderings can be accessed, yet.
\item RADICAL: Computes the radical of the ideal.
\item DIM: Computes the dimension of the ideal. So far I have no idea what the output means.
\end{enumerate}
\section{Problems}
Polymake allows negative exponents.

Singular allows different monomial orderings. So far we are only using one.

Rings should be able to have a grading other than the trivial one.
\end{document}
