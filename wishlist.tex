\documentclass[a4paper]{article}
\begin{document}
\section{Programming}
\begin{itemize}
	\item[] List Management
	\begin{itemize}
		\item[] Functions as in M2: all, select, unique, positions, apply
	\end{itemize}
\end{itemize}
All this can be done by perl, which is part of polymake.
\section{Mathematics}
\subsection{Combinatorics}
\begin{itemize}
	\item[] Fans
   \item[] secondary fan of a polytope (GFAN)
	\item[] Polyhedra
	\begin{itemize}
		\item[] Cones
		\item[] Polytopes
	\end{itemize}
\end{itemize}
Everything here can be solved by Polymake. Their interface to gfan needs to be updated.
\subsection{Algebra and Algebraic Geometry}
\begin{itemize}
   \item[] Ideals
	\item[] Divisors
	\item[] Modules
	\item[] Rings (local monomial ordering?)
	\item[] Varieties
\end{itemize}
We need to compute a standard basis for any ideal, as well as saturations and reduction of ring elements.
This can be done by singular.
\end{document}
