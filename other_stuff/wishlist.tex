\documentclass[a4paper]{article}
\begin{document}

\section{Objects we need (or would be happy to have)}
\begin{itemize}
\item Cones: 

There are two representations for cones, by generating rays and by hyperplanes. We need to switch between both.
We need to be able to compute a Hilbert basis for a given cone.
\item Divisors:

It would be nice to have divisors on affine, as well as on projective varieties. Additionally we would like to compute global sections of a divisor.
We would like to do the usual operations, like calculating the sum, etc.
We would like to determine, whether a divisor is semiample, ample, nef, big, etc., I am not sure for which of these there are algorithms.
\item Fans:

\item Graphs:

For quivers.
\item Groups:

We need abelian groups with torsion. Then we need to do the usual operations, like direct sum, Hom(), etc.
\item Ideals:

We need to have the standard operations, like intersection, saturation, sum, product, standard basis, groebner basis, etc.

\item Modules:

We need the standard operations, like Ext, Tor, $\otimes$, $\oplus$. We need to compute free resolutions and syzygies. The same in the graded cases.
\item Polyhedron:

We need to compute the vertices, the tailcone, the dimension, the d-dimensional faces, again switch between the two possible representations.
Also we would like to know the ambient dimension, whether the polyhedron is bounded/reflexive/full-dimensional/smooth/normal, the lattice points, etc.
We need to optimize linear functionals at polyhedra, and compute the minimal vertex/lattice point.
\item Rings:

Quotient rings, normalization, localization. Check whether two rings have the same quotient field.
\item Tropical Varieties:

Maybe not as a subclass of variety?
\item Varieties:

\end{itemize}
\section{Methods we need (...)}
Not all of the methods do make sense for every possible input, this is just to give some idea. Most of the methods are at first only needed for toric varieties, thus they can be easily be implemented in Polymake.
\begin{itemize}
\item normalFan(Polyhedron) = Fan 
\item Spec(Ring) = Variety, Proj(multigraded Ring) = Variety, TV(Fan) = Variety
\item globalSections(Divisor) = ?
\item dimension(Cone/Fan(?)/Variety/Ring/Polyhedron) = ZZ
\item secondaryFan(Polyhedron) = Fan
\item Cox(Variety) = Ring, Pic(Variety) = Group, Nef(Variey) = Cone, Eff(Variety) = Cone
\item etc.
\end{itemize}

\section{Programming}
We need the programming language to be very intuitive.
It would be nice to have a list management like in Macaulay2.
...
%It should be possible to write something like
%{\tt QQ[x,y,z]} without getting an error message. The system should then automatically specify the monomial ordering and the characteristic.
%Specifying the monomial ordering should be optional, the regular user would only be confused by having to do this.

%Also one should be able to specify new objects. This should be as intuitive as possible and stable. One should not have to contact the programmers in Kaiserslautern to declare a new object called p-divisor and simply made of two lists.

%For varieties/rings/modules/groups we also need maps between these objects. I think maps are easier to construct if one provides both source and target, instead of using {\tt setring} over and over.

%For error handling it would be very nice to include tests that are automatically run for every new version. It should also be checked which code really is accessed by such tests.
%\section{Projects}

%\subsection{Quivers vs. MDS}
%\subsection{The Hyperdeterminant}
%Compute the Chow quotient of the hyperdeterminant.
\end{document}

\section{Programming}
\begin{itemize}
	\item[] List Management
	\begin{itemize}
		\item[] Functions as in M2: all, select, unique, positions, apply
	\end{itemize}
\end{itemize}
All this can be done by perl, which is part of polymake. One should not have to use loops for all this, since one then would have to create new lists all the time, even if the old ones are going to be replaced. Also for select and positions, etc. faster search algorithms can be used. Hash Tables?
\section{Mathematics}
\subsection{Combinatorics}
\begin{itemize}
	\item[] Fans
   \item[] secondary fan of a polytope (GFAN)
	\item[] Polyhedra
	\begin{itemize}
		\item[] Cones
		\item[] Polytopes
	\end{itemize}
\end{itemize}
Everything here can be solved by Polymake. Their interface to gfan needs to be updated.

It would maybe be interesting to be able to interface to fast linear optimization solvers, like gurobi or cplex or syplex(?). Can these be used to build a faster version of Fourier Motzkin?
\subsection{Algebra and Algebraic Geometry}
\begin{itemize}
   \item[] Ideals
	\item[] Divisors
	\item[] Modules
	\item[] Rings (local monomial ordering?)
	\item[] Varieties: It would be nice to have functors like Proj and Spec. At the same time we need to be able to specify subvarieties of these to get Divisors.
\end{itemize}
We need to compute a standard basis for any ideal, as well as saturations and reduction of ring elements.
This can be done by singular.
\section{Projects}
\begin{itemize}
\item Implement the algorithm by Nathan for up and downgrading.
\item The hyperdeterminant problem.
\item Quivers of sections.
\end{itemize}


\end{document}
